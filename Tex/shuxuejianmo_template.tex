%! Tex program=XeLatex
\documentclass[12pt]{article}
\usepackage{geometry}
\usepackage{setspace}

\geometry{a4paper,left=2.5cm,right=2.5cm,top=2.5cm,bottom=2.5cm}
\renewcommand{\baselinestretch}{1.0}
\setlength{\parindent}{2em}

\usepackage{indentfirst}

%一级子标题居中效果
\usepackage[center]{titlesec}
\titleformat*{\subsection}{\raggedright\bf\large}
\titleformat*{\subsubsection}{\raggedright}

\usepackage{xeCJK}
\xeCJKsetup{AutoFakeBold=true,EmboldenFactor=1}
\setCJKmainfont{SimSun}
%\setCJKmainfont{Adobe Heiti Std}

\setCJKfamilyfont{song}{SimSun}
\newcommand*{\songti}{\CJKfamily{song}}

\setCJKfamilyfont{hei}{Adobe Heiti Std}
\newcommand*{\heiti}{\CJKfamily{hei}}

\usepackage{bm}


\begin{document}

\title{\centering\textbf{蒙气差的计算}}
\date{}
\maketitle{}

\begin{center}
  \textbf{摘要}
\end{center}

本文针对蒙气差的计算问题,利用相关的地理、热力学知识,相继构建了模型一:蒙气差与折射率模型、模型二:大气密度与高度模型,确定了蒙气差与大气高度之间的关系。最后针对不同观测点气温和气压的不同,确立了蒙气差的修正方法。

\newpage

\section{问题重述}
随着人们生活质量的不断提高,几乎每家都拥有了私家车,这也就造成了每

\section{模型假设}

\begin{enumerate}
  \item 乘客预先知道每种交通方式的效用,且会选择交通效用最大的路径;
  \item 高铁的开通与否对其他运输方式的票价、运行时间、候车时间等因素无影响;
\end{enumerate}

\newpage

\section{符号说明}

\begin{table}[!h]
\center
\setlength{\tabcolsep}{8mm}{}
\renewcommand\arraystretch{2.3}
\begin{tabular}{|c|cc|}
\hline
序号&符号&符号说明\\
\hline
1&$M_{k}$&第k种交通工具的票价\\
2&$R_{k}$&第k种交通工具的运价率\\
3&$L_{k}$&第k种交通工具的运行里程\\
4&$F_{k}$&第k种交通工具快速性的相对大小\\
5&$V(T)$&乘客的时间价值\\
\hline
\end{tabular}
\end{table}

\newpage

\section{问题一}

\subsection{北京-沈阳交通运输方式客运量分担率模型}

针对问题一,我们选取客运量分担率这一指标来对高速公路车辆通行压力进行研究。首先,建立各种运输工具的广义费用函数,并用Logit模型实现客运量分担率的计算;然后对高铁建成前后高速公路分担率这一指标进行比较,分析高铁开通对高速公路车辆通行压力的影响。

\subsubsection{各种交通工具的广义费用函数}

根据旅客在出行过程中的
行为选择分析,确定将经济性、快速性、方便性、舒适度、安全性
作为广义费用的衡量指标,并计算相应的广义费用函数。

\textbf{(一)安全性}

一直以来,安全都是乘客选择出行方式考虑的首要因素。不同的交通工具,其安全性的相对大小$S_{k}$不同。一般来说,铁路安全性最高,公路安全性最低。

\textbf{(二)经济性}

经济性也是乘客考虑的重要因素,尤其是需要自费出行时,各种运输方式对应的运费将直接影响乘客对交通工具的选择。我们用各种交通工具的票价$M_{k}$ 作为衡量其经济性的指标。

\section*{参考文献}

[1]李晓玉.基于logit模型对大西高铁客流分担率的预测[J].现代商业,2014(21):199-200.

[2]朱桃杏,任建新,张雪燕.基于效用函数的京津冀高铁旅游出行需求研究[J].铁道工程学报,2018,35(03):102-108
\end{document}
